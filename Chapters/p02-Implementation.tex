%************************************************
\chapter{Methodology}\label{p02:implementation}
%************************************************
When implementing the session tracking algorithm, a correct set of rules need to be defined (\eg, what user actions are part of the active session, how much time a user takes to read a certain text, how much time can a user read without translating, etc) in order to properly measure a working session.

Two language teachers, one teaching French and the second teaching Dutch, have been using the Zeeguu platform during their lessons, therefore they were the ideal candidates for discussing about how to properly measure their students' work. In section \ref{p02:interviews}, the structure of the interview is explained.

Once obtained the general outlines, the algorithm design was started. The algorithm was designed following the finite state machine approach, where the session lifespan transitions between different status until the closure is reached. The technical details of the algorithm are addressed in section \ref{p02:session_tracking}.

Finally, a dashboard was developed to enable teachers evaluate the results of the session tracking.


%TODO Some explanation about how this project was developed using a software engineering approach (SCRUM and software requirements and documentation)
