%************************************************
\chapter{Learning Analytics}\label{ch:learning analytics}
%************************************************

Learning Analytics is the research area focused on collecting, analyzing and visualizing information \cite{Brown2011} related to the learning activities within a learning platform. Most of the current research focuses on Learning Management Systems (LMS) such as Moodle (\url{https://moodle.org/}) \cite{Park2015} \cite{Romero2008} or Blackboard (\url{https://www.blackboard.com}) \cite{Arnold2012}. The existing research therefore focuses on systems that emulate a virtual university, which collects data about students, courses enrollments, assignments, exams and grades.

\paragraph{Learning Management System}

Is the application that enables and manages all the educational content. Typically it also controls courses registration and tracks data about the students' progress. Finally it provides reports based on the collected data \cite{Watson2007}.\\


The nature of Personalized Textbooks implies that the learning road is different for each student, therefore the information collected for one student might be very different from the rest (because of his subscriptions, the language being learned, the current level of knowledge, etc), making it more difficult to compare his learning performance against other students. Additionally, the level of granularity is more specific, both because is focused on a specific topic and language, and because the type of analysis goes into a deeper level of detail (\ie activity tracking  each individual reading or practice session).

In short, a new method for collecting and analyzing data is proposed.


