%************************************************
\chapter{Related work}\label{ch:learning analytics}
%************************************************

Learning Analytics is the research area focused on collecting, analyzing and visualizing information \cite{Brown2011} related to the learning activities within a learning platform. Most of the current research focuses on Learning Management Systems (LMS) such as Moodle (\url{https://moodle.org/}) \cite{Park2015} \cite{Romero2008} or Blackboard (\url{https://www.blackboard.com}) \cite{Arnold2012}. The existing research therefore focuses on systems that emulate a virtual university, which collects data about students, courses enrollments, assignments, exams and grades.

\paragraph{Learning Management System}

Is the application that enables and manages all the educational content. Typically it also controls courses registration and tracks data about the students' progress. It also provides reports based on the collected data \cite{Watson2007}.\\

Previous studies focus on analyzing grades and overall performance for students in different courses and within a compared them with the rest of the class. While the "personal" nature of Zeeguu, implies that the information should be tracked for a specific article in a particular language. 

Personalized Textbook is a didactic solution that uses personal preferences to provide interesting reading texts and exercises based on those texts. This implies that:
\begin{itemize}
	\item the learning material is not the same for every student,
	\item the translated words are individual,
	\item the practiced vocabulary is completely unique and
	\item the language(s) being learned can differ.
\end{itemize}

Therefore:
\begin{itemize}
	\item The information collected for one user might not be the same as for the rest.
	\item The learning road is different for each student.
	\item It is more difficult to compare a student's learning performance against others. 
\end{itemize}

This project focuses on tracking each individual reading and practice session, to provide a detailed usage information. In consequence, a new method for collecting and analyzing learning time is required.


