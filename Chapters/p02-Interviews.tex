%************************************************
\chapter{Interviews}\label{p02:interviews}
%************************************************

During the first pilot test of the Zeeguu system, users demonstrated an interest in continue using the system. However, the two teachers that used the platform concluded that is difficult to keep track of the time spent by the students. One of them decided to ask his class a weekly report mentioning the amount of time and the read articles. The second one, decided to evaluate only the words that they translate, since this was the only available information.

Same goes for exercises, the results can be given but not how much time was spent by the students reinforcing their knowledge.

During the interviews, both teachers mentioned the need of precisely measuring the reading and practice time. The first teacher grants his students points for every half hour of work, which puts an extra pressure on the students, and sometimes they reported more reading time than real. This happens specially when students are reaching a deadline and have not made enough work. 

One teacher also evaluates the amount of work every week and gives them a classification (\Ie green, orange and red), which enforces the commitment of students.

A transcript of the interviews can be found in the Appendix section (\ref{pt:appendix}).