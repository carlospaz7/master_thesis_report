%************************************************
\chapter{Dashboard}\label{p02:dashboard}
%************************************************

As mentioned earlier, learning analytics is about providing information to the user so that they can reflect about their own activity and modify their actions to better achieve the expected results. For this reason a visualization of the session tracking is implemented.

By analyzing the feedback provided by the teachers during the interviews, they expressed that they would like to view the activity of their class but mostly at the individual level, given the personal nature of the Zeeguu platform.

A simple dashboard with a heat map was designed, providing a visual overview of the class activity in a single screen. The teacher is therefore able to view which students are active and when, what type of activity they do and how much. A heat map was chosen because is an easy to interpret visualization that does not require any particular expertise (figure \ref{fig:heat_map}).

%TODO: add picture of heat map

Additionally, a detailed view of how individual work is decomposed is provided by means of simple bars and percentages that give the teacher information about how much time was spent on reading and how much on exercises (figure \ref{fig:student_activity}).

%TODO: add picture of student activity details 

Interactive functionalities are also included, so that the teacher can analyze the class on his own and make particular discoveries (\Eg what students are falling behind, who has not worked in the last week, etc).


%TODO Explain material design