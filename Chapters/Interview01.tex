%************************************************
\chapter{Interview A}\label{interview01}
%************************************************
\begin{itemize}

	\item \textit{Question: What kind of information would you like to get and what is the purpose?}

I give a lot of freedom to my students to choose their activities. I don't mind what they do as long as they are reading. That is a very modern approach to language teaching so they are exposed a lot to the language, and the only thing I want is that they spend for example half an hour a day or an hour a week on reading, and I award them with points for every half hour of work. They have to get 60 points a year for example.
They get to choose for reading activities, speaking activities or writing activities. When it comes to reading activities that is when Zeeguu comes in. A number of students are very fond of Zeeguu, they don't do anything else for reading, so they spend a couple of hours or minutes on the system reading, checking for translations, doing exercises. My students are very motivated but even motivated students, when the deadline approaches tend to become less motivated to do the things right away. They tend to become a bit more easy on the honesty, so when the deadline approaches they might be doing something during 10 minutes and they report on the paper that they spent 60 minutes. I say to my students I cannot look into your head, I cannot see what's happening in your head, I cannot see what you're doing in terms of language acquisition the only thing I can see is that you spent an amount of time on reading activities. That is where this dashboard might be a little more helpful than now, because now I only have their words, so they fill out the report saying for example that they spend 30 minutes or 60 minutes on a certain text I can look into the system and I can see the difficulty of the text and I can see how many words they have asked for a translation but that's all. So what I need is more things that give me the idea that they really worked for example 30 minutes. I need to be convinced that they are doing that amount of time. 
I don't see anything on their activities like their exercises, the time if that is possible. 
I don't think it's fair when I don't convince my students that I have serious instruments that can make a positive guess about their work. When I don't do that I leave it to their self-discipline and I don't think it's fair to the students. They need to have some kind of guidance, so I have to put a little pressure on them to do the right thing.

	\item \textit{Question: So that reinforces the idea that both you and your students can see the same information and they can realize that this is what the professor is going to visualize, so maybe I have to put a little extra effort?}
	
Yeah that would be very nice.

	\item \textit{Question: Talking about the weekly report that you mentioned, you only ask them how many hours they worked? }

They fill out of this form telling me their name, the classroom, the date that they did this activity, the text title, the magazine they chose the text from and how many minutes, 30 or 60 minutes.

	\item \textit{Question: I'm just trying to understand the purpose behind the report, one side is for keep track of what they are doing but is it also for motivating?}
	
Yes it's external motivation, because most of students don't like to do homework when they are home they want to do other things. I have to put pressure on them to do the things they really should do in order to get to the exam. so I have chosen for a system in which I can rely on them to do things at home by giving them 60 points for 30 hours, that means about one hour a week, and as a personalized system they can choose whatever activity they want to do. In the end, when they don't have the 60 points they can't go to the next year.

	\item \textit{Question: The next question is how will you measure the quality of their work? Would it be only the amount of time do they spend reading? Or as well the score on the exercises? Or do you have another way of evaluating the quality?}
	
Now I only have quantitative measures I can look at the text, I can look at what words they have been asking to be translated and that gives me an idea about the quality of their work. Sometimes I see that they ask for too easy words and I talk to them about it. Whenever I see something that I don't like I write down the names and after class I ask them to come, and I ask them "why did you look up that word? You know this word, or at least I think you should know this word". From time to time, I have a discussion with those students to probe quality, but that is difficult because when I say "I think you have learnt these words in the first class 4 years ago", they can say "well, I forgot" and there is nothing I can do about it. I keep asking those questions and when I ask those questions that is a way I put pressure, they know that if they take it too easy I can ask a lot of questions that they don't like so they are triggered to do their best I think.
But it might be helpful indeed to have some information about how they performed in those exercises.

	\item \textit{Question: And after how many correct exercises of a specific word would you say that they have learned the word? We can ask them ten times the same word and maybe after the fifth time they have learned the world already.}
	
That is a very difficult question; the basis of the approach that I use is statistical learning. They see words, they see texts, they see language, and they make inductions about that, and every time whenever they see it for the second, third, fourth time it is enhanced, that meaning is best kept in mind, so every repetition is welcome.
It's a bit compared to when you learn your mother tongue you have a lot of exposure to the language, the amount of exposure is so enormous to the first language and I don't get that the amount of exposure with my students because I only have three classes a week.

	\item \textit{Question:Are you interested in analyzing individual students, or as a class or both?}
	
Individual in the first place because it's personalized, but to have information about how a class performs might be helpful as well.
I have for example two classes one of them is very loyal, always have the points on time and the other group doesn't. That is something that I see and this will be reinforced by information; but for me it's not necessary because I evaluate them on the individual level. 
And the second problem is that not every student uses Zeeguu to get his points.

	\item \textit{Question:Another question will be what level of detail are you interested? For example at the word level or a bit more general for example at the session level, and how many sessions or how much time have they spent? }
	
The amount of time they spend on the system is very important, I think. 
And the amount of hours they do during the exercises might be a good indicator of their effort. When they are having a lot of errors, they are not having a mental activity they are just choosing whatever. When they are busy reading they are busy with meaning. 

	\item \textit{Question:How often did you evaluate their work? Is it a daily, weekly, monthly?}
	
Two monthly. I have five times a year so I have 5 deadlines and that is in terms of 2 months. I asked for example 12 points by the end of the first two months. And if they don't, they get an application with me to do two points a week. I work with levels: orange, green and red so when you are on level green you're ok, you can choose whatever you want, and when you don't meet my demands you go to level orange. That means every week you have to produce two points. And after the second time that they don't reach the deadline, they go to level red.

	\item \textit{Question: This question is related to analyzing the students as a class. Are you interested in detecting special students like the ones that are working a lot?}
	
It doesn't tell me much because some students don't use the system. What might be interesting is that the students might be able to see the ranking of the articles. The ones that they like a lot. Maybe that can be done in the system. It saves a lot of time to them because I have to look at the text and decide which one to take. And with a lot of extra information from other students, it might save time. 

	\item \textit{Question:Is there something that you would like to add?}
	
Normally teachers give homework that is not personalized. Students have text in the book for example or text from the internet that is copied and they work with that. What might be a good addition for these colleges for example is that the teacher chooses a text and designs a sort of test to be able to verify that they have done the work and that is of quality. I am not really interested in quality but in quantity.
\end{itemize}