%************************************************
\chapter{Dashboards}
%************************************************

The term "dashboard" has evolved over time, the latest definition refers to a visualization of the most important information in order to achieve a particular goal, and within the visual space of a computer screen \cite{Park2015}.

The main goal of a dashboard is to provide timely and useful information so that an decision can be made consequently. \citeauthor{Verbert2013} describe the mental process that an user of a dashboard performs to make sense out of it. He states that a good dashboard should first help users acquire self-awareness by looking at data about their past activity and their current state in the learning process. Second, it invites to do a personal reflection. So that these questions get answered by exploring the visualizations and finally it must produce a change of behavior based on newly set goals. \citeauthor{Telea2017} describes the visualization design pipeline as an iterative process where the goal is to narrow the scope of the problem until a clear question can be answered with a visual tool. The visualization has to be designed for a specific user in mind, taking into account the complexity for interpreting the visualization and the level of precision when decoding the information back into abstract data.

A good quality dashboard should provide all the relevant information in a summarized way to enable the user solve a particular problem. Two set of principles were developed to evaluate dashboards design. 
\begin{itemize}
	\item SMART (synergetic, monitor KPIs, accurate, responsive,
	and timely)
	\item IMPACT (interactive, more data history, personalized, analytical, collaborative, and traceability) \cite{Malik2005}\\
	
\end{itemize}

When creating a new dashboard, the most relevant information should be located at the center so that it stands out, and with the remaining space around the central visualization, it should provide details that support the answers obtained with the central element.

