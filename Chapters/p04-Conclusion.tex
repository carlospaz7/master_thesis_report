%************************************************
\chapter{Conclusion}\label{p05:conclusion}
%************************************************

As observed in Chapter \ref{p04:discussion}, not only the time spent but also the work patterns and habits can be spotted with a precise session tracking. Students who work uninterruptedly versus those who lost focus very frequently can be easily detected (Figure \ref{fig:Exercise_timeline}). Then, teachers can use multiple approaches to evaluate the progress of students in the individual level and also as a class.

The extensibility and applicability of the session tracking algorithm is vast. The proposed algorithm can be used for other purpose ITS, by only adjusting the timeout and the particular set of rules for the specific application. Outside of the learning context, the session tracking algorithm could also be used to track customers engagement with a particular website, and interesting questions can be crafted and answered via visualizations. In the office, session tracking can also be implemented to detect employees who are having too much distraction from their activities. 

For our particular application, the Zeeguu platform, one of the main goals was to keep students from gaming the system. With that goal in mind the event tracking and the timeout parameters had to be strict. However, not for every application must be the same.

Finding the optimal timeout value is not a trivial task, it requires sample data and multiple tests to find a good setting, and even then, it can vary from context to context (\eg, depending on the level of expertise of users, on the type of ITS, on the level of trust that can be placed on users, etc).

