%************************************************
\chapter{Conclusion}
%************************************************

As observed in section \ref{p03:results}, not only the time spent but also the quality of the work can be assessed with a precise session tracking. A teacher can use multiple approaches to evaluate the progress of students in the individual level and also as a class.

The extensibility and applicability of the session tracking algorithm is vast. The proposed algorithm can be used for other purpose ITS, by only adjusting the timeout and the particular set of rules for the specific application. As well as for the commercial industry, the session tracking algorithm can be used to track customers engagement with a particular website, and interesting questions can be crafted and answered via visualizations. In the office, session tracking can also be implemented to detect employees who are having too much distraction from their activities. 

For our particular application, the Zeeguu platform, one of the main goals was to keep students from gaming the system. With that goal in mind the event tracking and the timeout parameters had to be strict. However, not for every application must be the same.

Finding the optimal timeout value is not a trivial task, it requires sample data and multiple tests to find a good setting, and even then, it can vary from context to context (\Eg depending on the level of expertise of users, on the type of ITS, on the level of trust that can be placed on users, etc).

