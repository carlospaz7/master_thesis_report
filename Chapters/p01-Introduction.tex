%************************************************
\chapter{Introduction}\label{ch:introduction}
%************************************************

\section{Introduction}
ITS (Intelligent tutoring systems) are applications that provide a user whose learning a topic, with adaptive capabilities that evaluate the progress of his learning process and provide custom feedback that helps the student focus on his opportunity areas.
ITS are also used by teachers who monitor the students’ progress and help them provide timely and personalized feedback to the students during their learning process.
\cite{Jugo2014}, \cite{Romero2008} and \cite{Dogan2009} have used data mining algorithms to extract relevant patterns in the learning process. Ranging from normal statistical visualizations \cite{Romero2008} to complex multidimensional visualization \cite{Dogan2009}, different approaches to provide understandable insight into the learning process have been researched. \citeauthor{Verbert2013}\cite{Verbert2013} has performed a comprehensive comparison of different analytical dashboard for ITS systems. Gamification \cite{Gonzalez2014} has also been proposed as a way to tackle lack of interest due to monotony in the learning experience.

\section{Research questions}
With the dynamic capabilities of the Web, a personalized textbook for learning foreign languages \cite{Mircea2018} is an ongoing research project. The tool allows users to read and learn new vocabulary by recommending texts of their own interest. A pilot test was implemented in a classroom and both students and professors showed an interest in using a tool like the one proposed. One of the outcomes from this research was that professors wanted to have monitoring functionalities that would allow them to keep track of their students' work and progress.
Therefore an interactive dashboard with a set of relevant metrics was implemented.
The two main research questions that get answered are:

\begin{enumerate}
\item As a user of the Zeeguu System I want to see my learning progress in the system.
\item As a teacher I want to see my students' usage of the system, as well as the amount and quality of work.
\end{enumerate}


\section{Significance of research}
Personalized textbook project \cite{Mircea2018} proved to be an auxiliary tool in the learning process by providing dynamic learning material and exercises according to the level of knowledge and the interests of the student. However, for the teacher it is difficult to evaluate the progress done by the students, so they need to ask for a written report about the work performed, which is not optimal nor trustworthy source of information. 
By enabling the teachers to monitor in real time the work done by their students and the difficulties they are having, teachers are given additional tools so that they can personalized teaching assistance according to the students’ needs.
Furthermore, with the continuous appearance of learning platforms and accessibility to dynamic content, it is certain that in the following years, new teaching methods that use technologies like this will emerge.
These metrics will be extensible to other learnings platforms and topics.


