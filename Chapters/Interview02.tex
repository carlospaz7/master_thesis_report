%************************************************
\chapter{Interview B}\label{interview02}
%************************************************
\begin{itemize}
	
		\item \textit{Question: As a teacher, what would you like to visualize or to understand from the system?}
		
I would like to know more or less what kind of resources they are using and what are the kind of articles they are reading. Let's say at this moment that a student reads a text about the sun and another student read that text about the same topic in the Guardian. It will be great to have a visualization on the resources, I want my students to read The Guardian for instance, but I cannot force them to do so, but it will be great to have a clearer view of the resources they use, and if possible the kind of article.
imagine that these articles are already categorized according to the newspaper, so one article is about politics, and others about sports or something, it would be great to have something as well in the tool. So that you can easily say, I want you to read a text about sports but you didn't because you read something about politics.
So some categorization of text types and resources. 
The time students spend actually reading, that is I think already automatic in Firefox you have the reading tool in Firefox and it automatically says how much time you should spend approximately on reading that article. Let's say you are reading an article in the Guardian and it says, well it's approximately 6 minutes. But then it would be great to have it measured if it's really six or seven minutes.
It will be great to see how many words a student looks for. Imagine that you have a text of 200 words and that student searches for 150 words, then that information could help you help the student.
The amount of time students spend on the exercises, because now it says two minutes three minutes or five minutes which will take approximately.
You can more or less say you have one hour for that week and that means that I have 12 texts of 5 minutes and I want you to read this texts.
And the difficulty of the text, so if a student reads only the more simple texts then you'll see it's not improving, you should more or less see some kind of improvement over time. 
Now the tool is very user centered, so is a student who reads text autonomously, and as a teacher, you could say ok I want you to read so many minutes, so that the system collects this information.
Eventually it might also be interesting to be able to limit the amount of texts to say I want you to do this or that, as a teacher. 

	\item \textit{Question: How would you measure the quality of the work?}
	
It depends on the sources, on the text length (the readability score). I see the activities afterwards like exercises more as a way to practice, a fun way to work with words. But I find a text more important, the text complexity and the way they interact with text. I see the activities afterwards as an attractive way to spend a little more time on the words that I looked up but it's not the best way, you can also use a dictionary or write the words in their own dictionary. I personally feel that they have to do more with the words. You can also say I just want to improve my receptive word knowledge which is also great, but on the other hand, you also want to use these words, that is quite complex to interact with a text that you would normally not interact with. That would also be interesting to see how to force people to interact with the text. I could be doing a summary or giving an opinion. 

	\item \textit{Question: Is it important for you to understand each student individually or as a class?}
	
Preferably, each student individually, but it would also be great to have an overview of the class as a group. The common European framework for reference it's a bible for people who are familiar with the language portfolios, Zeeguu can be seen as a repository for demonstrating your reading knowledge. You are collecting online text and that tells about your reading knowledge. Therefore you can then say that a text is for instance at an A2 level, but imagine a text about environmental problems in America will be more at a B2 or C1 level. So that means that I am improving myself, so eventually I am in this level. 

	\item \textit{Question: How would you categorize a text according to a different level in the framework?}
	
According to the framework you can only read texts from B1 and so on. But then again that is not entirely true because you can use translation tools in Zeeguu, so Zeeguu allows you to read texts even if you are not B1 or B2. 
Imagine that you have a frequency, and 80\% of the words of a text is in the range from 0 to 2000, so if you now these words, which you should already know at a A2 level, you should be able to search for the difficult words in the higher range meaning that you can basically read an authentic text because you already know the other kinds of words, and there is information about the frequency list.

	\item \textit{Question: How would you classify a word as known or unknown? If you find the text and you don't translate any word do we infer that they know all the words?}
	
That is quite difficult so maybe you should try to avoid that question, that is something that we discussed about, the question is that if they improve overtime? That is really difficult to answer. For a learner if you use the tool and it works for you, that's great. And as a teacher you can ask your students to use the tool to improve their language, but maybe that's all.

	\item \textit{Question: When you evaluate your students, what kind of time window do you use?}
	
Probably weekly.

	\item \textit{Question: Is it important for you to detect special cases of students? Like outliers from the normal class? Or do you prefer to analyze them individually?}
	
It is quite difficult what you propose, imagine a student that reads only texts from the sun and does not look up for any word, and you have students who read text from the Guardian who look up for 20 words because they have intrinsic motivation to improve their words, so it is difficult to compare different articles.

	\item \textit{Question: With your current experience with the system, how do you evaluate your students? Or their progress?}
	
I see my students every week so I talk to them about the tool, and I made it obligatory for them to use the tool in order to improve their reading knowledge. I have my reader with seven thousands words, they are forced to study these words study these words, I have the same approach. I have to first look into the type of word (is it a verb, or a noun, or an adjective and so on), they probably have to look up for some words and then they have to come up with ten questions that they have to ask the other students, and they have these words in brackets as a stimulus for response. So I want them to use these words in different circumstances. That is explicit, then Zeeguu as a fun way to learn words implicitly, so I have two different approaches to vocabulary.

	\item \textit{Question: Did I miss something that you consider relevant?}
	
It would be great to have more information about time that students spend, about the reading time, text complexity, resources and perhaps also text topic.
I was just wondering if it would be possible to select texts based on difficulty as well, according to the measurement tool that you are using right now. Imagine that one wants to read only texts with a difficulty of 8 or above in the selected languages, because less than 8 is supposedly too easy.
\end{itemize}